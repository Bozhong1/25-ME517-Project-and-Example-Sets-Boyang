\section*{Project II: Literature Review (due Oct 3)}

The second step in your semester-long research proposal development is to contextualize your problem within the current field of your choice, demonstrate your understanding of the state-of-the-art, and identify something we don't yet understand but need to---this is the gap your (hypothetical) proposed project would seek to fill.

You'll execute this portion of the project as an outline. 
This outline does not need to be long! 
It does, however, need to be very clear, as you'll be expanding on it later. 
The sections should be the following:

\renewcommand{\outlinei}{enumerate}
\renewcommand{\outlineii}{itemize}
\begin{outline}
    \1 \textbf{Introductory context}
        \2 Add one or two bullet points briefly framing the history of your topic and its significance. 
        \2 Citations here are important---there should be several well-placed citations in each of these bullet points, and good review papers are especially helpful to lean on for context. 
        \2 \textit{\textbf{Example}: While gradient materials occur widely in nature---the structurally protective gradient of squid beaks [1], junctions between ligaments and bones [2], and the byssus threads that hold mussels to rocks [3] to name a few---polymeric gradient materials were only first considered in an engineering context in 1972 [4].}
    \1 \textbf{The state of the field}
        \2 In one or two bullet points, explain how things are done in the area of your project at the present moment. 
        \2 This can include any of experimentation, computational methods, and/or theory.
        \2 \textit{\textbf{Example 1}: Compositional gradients in engineered materials are typically produced in one of three ways: spatially (1) varying polymer crosslink density, e.g. using ultraviolet light-sensitive reactive groups [5, 6], (2) seeding of micro- or nanoparticles using centrifugation [7], electric fields [8], or other methods [9], or (3) using porosity to create structural gradients using dissolvable template-making materials [10].}
        \2 \textit{\textbf{Example 2}: Additive manufacturing has emerged as perhaps the best option for making complex functionally gradient soft materials on the order of cm or larger [11-14].}
    \1 \textbf{The Big Gap}
        \2 In one or two sentences, what is it that we don't know, and why isn't it solved? 
        \2 \textit{\textbf{Example 1:} The principal limitation of the first three techniques is scalability to larger sizes.}
        \2 \textit{\textbf{Example 2:} Arguably, the most challenging barrier for widespread production of gradient materials is the combination of sample repeatability and a comprehensive lack of validation options.} 
\end{outline}

\textit{A strong literature review outline will succinctly illustrate the essential context of the problem, the current best knowledge in the area, and the critical gap in knowledge restricting further advances or implementation, and should cite approximately 10-15 references with a \LaTeX ~bibliography.}


%This is just a placeholder for now


